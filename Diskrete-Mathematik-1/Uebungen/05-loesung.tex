\documentclass[12pt.twoside,a4paper,notitlepage,parskip]{scrartcl}
\usepackage[utf8x]{inputenc}
\usepackage{ucs}
\usepackage{ngerman}
\usepackage{amsmath}
\usepackage{amsfonts}
\usepackage{amssymb}
\usepackage{nameref}
\usepackage{enumerate}
\usepackage{graphicx}

\begin{document}
\title{Diskrete Mathematik I, WS 2011, Übung 5}
\author{
\begin{tabular}{cc}
Tilman Bender   & Thomas Tacke \\
108011247244   & 108011267882 \\
\end{tabular}
}
\date{\today}
\maketitle

\section*{5.1}
\subsection*{a)}
\begin{enumerate}
\item $v=5, t=3, V = V  \setminus \{5\}, E = E \setminus \{(3,5)\}$
\item $v=6, t=32, V = V \setminus \{6\}, E = E \setminus \{(2,6)\}$
\item $v=2, t=321, V = V \setminus\{2\}, E = E \setminus \{(1,2)\}$
\item $v=1, t=3213, V = V \setminus \{1\}, E = E \setminus \{(1,3)\}$
\item $v=3, t=32134, V=V\setminus \{3\}, E=E\setminus \{(3,4)\}$
\item $v=7, t=321344, V=V\setminus\{7\}, E=E\setminus \{(7,4)\}$
\end{enumerate}
\underline{t=321344}

\subsection*{b)}
\begin{enumerate}
\item $i=1, S=\emptyset , t_{1\dots6}=272537 \rightarrow s_{1}=1, e_{1}=\{1,2\}$
\item $i=1, S=\{1\} , t_{2\dots6}=72537 \rightarrow s_{2}=2, e_{2}=\{4,7\}$
\item $i=1, S=\{1,2\} , t_{3\dots6}=2537 \rightarrow s_{3}=3, e_{3}=\{6,2\}$
\item $i=1, S=\{1,2,3\} , t_{4\dots6}=537 \rightarrow s_{4}=4, e_{4}=\{2,5\}$
\item $i=1, S=\{1,2,3,4\} , t_{5\dots6}=37 \rightarrow s_{5}=5, e_{5}=\{5,3\}$
\item $i=1, S=\{1,2,3,4,5\} , t_{6\dots6}=7 \rightarrow s_{6}=6, e_{6}=\{3,7\}$
\end{enumerate}

\begin{figure}[htbp]
\begin{center}
\includegraphics[scale=0.5]{A51b}
\caption{Graph für Prüfercode 272537}
\label{default}
\end{center}
\end{figure}


\section*{5.2}
\subsection*{a)}
Siehe Abbildung \ref{fig:52a} auf Seite  \pageref{fig:52a}
\begin{figure}[htbp]
\begin{center}
\includegraphics[scale=0.5]{A52a}
\caption{Digraph für Aufgabe 5.2a}
\label{fig:52a}
\end{center}
\end{figure}

\subsection*{b)}
Starke Zusammenhangskomponenten des Graphen sind die Kreise  (1,2,3,4), (3,4), (5,6,8)

\subsection*{c)}
G ist nicht azyklisch. Wie bereits unter b) angegeben enthält der Graph mehrere gerichtete Kreise.

\section*{5.3}
\begin{align*}
P_{k=0}=
\begin{pmatrix}
	0 & 0 & 0 & 0 & 0\\
	0 & 0 & 0 & 1 & 0\\
	1 & 1 & 0 & 0 & 0\\
	0 & 0 & 1 & 0 & 0\\
	0 & 0 & 0 & 1 & 0\\
\end{pmatrix}
P_{k=1}=
\begin{pmatrix}
	0 & 0 & 0 & 0 & 0\\
	0 & 0 & 0 & 1 & 0\\
	1 & 1 & 0 & 0 & 0\\
	0 & 0 & 1 & 0 & 0\\
	0 & 0 & 0 & 1 & 0\\
\end{pmatrix}
P_{k=2}=
\begin{pmatrix}	
	0 & 0 & 0 & 0 & 0\\
	0 & 0 & 0 & 1 & 0\\
	1 & 1 & 0 & 1 & 0\\
	0 & 0 & 1 & 0 & 0\\
	0 & 0 & 0 & 1 & 0\\
\end{pmatrix}\\
P_{k=3}=
\begin{pmatrix}
	0 & 0 & 0 & 0 & 0\\
	0 & 0 & 0 & 1 & 0\\
	1 & 1 & 0 & 1 & 0\\
	1 & 1 & 1 & 1 & 0\\
	0 & 0 & 0 & 1 & 0\\
\end{pmatrix}
P_{k=4}=
\begin{pmatrix}
	0 & 0 & 0 & 0 & 0\\
	1 & 1 & 1 & 1 & 0\\
	1 & 1 & 1 & 1 & 0\\
	1 & 1 & 1 & 1 & 0\\
	1 & 1 & 1 & 1 & 0\\
\end{pmatrix}
P_{k=5}=
\begin{pmatrix}
	0 & 0 & 0 & 0 & 0\\
	1 & 1 & 1 & 1 & 0\\
	1 & 1 & 1 & 1 & 0\\
	1 & 1 & 1 & 1 & 0\\
	1 & 1 & 1 & 1 & 0\\
\end{pmatrix}
\end{align*}
\section*{5.4}
\subsection*{a)}
\begin{table}[htdp]
\caption{default}
\begin{center}
\begin{tabular}{|l|l|}
	pred & Q \\
	 $[\infty,\infty,\infty,\infty,\infty,\infty,\infty,]$ &  \\
	 $[\infty,1,\infty,\infty,\infty,\infty,\infty,]$ & 2 \\
	 $[\infty,1,1,\infty,\infty,\infty,\infty,]$ & 23 \\
	$ [\infty,1,1,1,\infty,\infty,\infty,]$ & 234 \\
	 $[\infty,1,1,1,\infty,\infty,2]$ & 347 \\
	$[\infty,1,1,1,\infty,3,2]$ & 476  \\
	$[\infty,1,1,1,4,3,2]$ & 765  \\
	$[\infty,1,1,1,4,3,2]$ & 65  \\
	$[\infty,1,1,1,4,3,2]$ & 5  \\
	$[\infty,1,1,1,4,3,2]$ &   \\
\end{tabular}
\end{center}
\label{default}
\end{table}%

\subsection*{b)}
\begin{table}[htdp]
\caption{default}
\begin{center}
\begin{tabular}{|c|l|l|}
	v &pred & S \\
	1 &$[nil,nil,nil,nil,nil,nil,nil]$ &  \\
	2 &$[nil,1,nil,nil,nil,nil,nil]$ & 1 \\
	4 &$[nil,1,nil,2,nil,nil,nil]$ &  12\\ 
	5 &$[nil,1,nil,2,4,5,nil]$ &  124\\
	6 &$[nil,1,nil,2,4,5,nil]$ &  1245\\
	5 &$[nil,1,nil,2,4,5,nil]$ &  124\\
	4 &$[nil,1,nil,2,4,5,2]$ &   12\\
	2&$[nil,1,nil,2,4,5,2] ]$ &   12\\
	7&$[nil,1,nil,2,4,5,2]$  &   1\\
	2&$[nil,1,1,2,4,5,2]$  &   ?\\
	1&$[nil,1,1,2,4,5,2]$  &   1\\
	3&$[nil,1,1,2,4,5,2]$  &   ?\\
	6&$[nil,1,1,2,4,5,2]$  &   ?\\
	3&$[nil,1,1,2,4,5,2]$  &   ?\\
\end{tabular}
\end{center}
\label{default}
\end{table}%

\end{document}
