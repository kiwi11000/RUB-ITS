\documentclass[12pt.twoside,a4paper,notitlepage]{article}
\usepackage[utf8x]{inputenc}
\usepackage{ucs}
\usepackage{ngerman}
\usepackage{amsmath}
\usepackage{amsfonts}
\usepackage{amssymb}
\usepackage{nameref}
\usepackage{enumerate}

\begin{document}
\title{Netzsicherheit I, WS 2011, Übung 2}
\author{
\begin{tabular}{ccc}
Tilman Bender & Christian Kröger & Thomas Tacke \\
108011247244 & 108011250663 & 108011267882 \\
\end{tabular}
}
\date{\today}
\maketitle

\section*{Aufgabe 3.1}
\begin{itemize}
\item Annahme: DieSchuler stammen nicht aus einer Klonarmee.\\
\item Urnenmodell
\begin{itemize}
\item Sportkurse = Urnen
\item Schüler = Bälle
\end{itemize}
\end{itemize}
\begin{enumerate}[a)]
\item Beliebig viele in jedem kurs. 
\item Mindestens einer an jedem kurs.
\item maximal einer in jedem kurs.
\end{enumerate}

\section*{Aufgabe 3.2}
\begin{enumerate}
\item 
\item
\end{enumerate}

\section*{Aufgabe 3.3}



\end{document}
