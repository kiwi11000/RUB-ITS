\documentclass[12pt.twoside,a4paper,notitlepage,parskip]{scrartcl}
\usepackage[utf8x]{inputenc}
\usepackage{ucs}
\usepackage{ngerman}
\usepackage{amsmath}
\usepackage{amsfonts}
\usepackage{amssymb}
\usepackage{nameref}
\usepackage{enumerate}

\begin{document}
\title{Netzsicherheit I, WS 2011, Übung 4}
\author{
\begin{tabular}{ccc}
Tilman Bender & Christian Kröger & Thomas Tacke \\
108011247244 & 108011250663 & 108011267882 \\
\end{tabular}
}
\date{\today}
\maketitle

\section*{a)}
Eine der Herausforderungen beim Pay-TV ist das Schlüsselmanagement für eine sehr große und sich ändernde Anzahl von Nutzern. Des weiteren ist das Vertrauensmodell beim Pay-TV ein anderes als bei anderen kryptographischen Systemen. 

In anderen kryptographischen Systemen haben die Nutzer ein berechtigtes Eigeninteresse den Schlüssel geheim zu halten damit ihnen kein Nachteil entsteht. Bei Pay-TV entsteht dem einzelnen Benutzer kein Nachteil, wenn er seinen Schlüssel preisgibt. Lediglich der Anbieter hat wirtschaftliches Interesse daran den Schlüssel geheim zu halten. 


Wie groß die Motivation der Angreifer unter den Nutzern ist, zeigt die Tatsache, dass eines der frühesten Beispiele für Seitenkanalangriffe im Bereich Pay-TV veröffentlicht wurde.

\section*{b)}
Da aus Sicht des Pay-TV Anbieters jeder Nutzer ein potentieller Angreifer ist und der Nutzer die Hardware und evtl. Software auf Seite des Decoders kontrolliert, stellt die Chipkarte die minimal vertrauenswürdige Umgebung dar. Nur dort  kann die Überprüfung der Rechte sicher stattfinden. Schlägt diese Überprüfung fehl, so wird das Kontrollwort in der Chipkarte verworfen, der Film bleibt verschlüsselt.

\section*{c)}
Die Entitlement Management Message (EMM) dient quasi zur Vergabe von Berechtigungen. Sie enthält den mit dem Private Key (PK) verschlüsselten Programmschlüssel (Service Key, SK)  sowie die Berechtigungen des Benutzers. Die ECM ist mit einem MAC vor Manipulationen geschützt. Auf der Chipkarte wird mit Hilfe des PK die EMM entschlüsselt um den SK zu erhalten. Der SK wird anschließend verwendet um aus der ECM das Kontrollwort zu entschlüsseln.

Die Entitlement Control Message (ECM) hingegen dient zur Kontrolle der Berechtigungen. Sie enthält das mit dem Programmschlüssel (SK) verschlüsselte Kontrollwort sowie zusätzliche Bedingungen (Jugendschutz, Abonnements etc.), die es zu überprüfen gilt, bevor dem Nutzer Zugriff gegeben wird. Die ECM ist mit einem MAC vor Manipulationen geschützt. 

\section*{d)}
Der Pay-TV-Anbieter verschlüsselt und authentisieret die Programmschlüssel SK und die Rechte des Teilnehmers mit dessen persönlichem Schlüssel  und überträgt diese in der Entitlement Management Message (EMM) an die Chipkarte des Teilnehmers.

Der Decoder filtert die ECMs aus dem allgemeinen Datenstrom  und leitet sie an die Chipkarte weiter . Die Chipkarte Überprüft MAC  und entschlüsselt die Daten mit dem zuvor übertragenen SK. Danach werden die in der ECM enthaltenen Bedingungen mit den auf der Chipkarte gespeicherten Rechten verglichen. Stimmen  die Rechte mit den jeweiligen Bedingungen überein, wird das Kontrollwort von der Chipkarte an den Decoder weitergeleitet und das Videosignal wird entschlüsselt/descrambeld.

\section*{e)}
Ein Problem bei der Verteilung des Programmschlüssels besteht darin, dass die Berechtigung eines Nutzers erlöschen kann. Beispiel: Die Kunden eines Anbieters können zum Monatsende kündigen. Ein bisher berechtigter Nutzer A hat bereits den gültigen SK. Somit muss der Anbieter, nachdem A gekündigt hat, einen neuen SK erzeugen und ihn an alle anderen noch berechtigten Teilnehmer verteilen. Diese können den neuen Schlüssel nur empfangen, wenn ihre Geräte auch eingeschaltet sind. Der neue Schlüssel müsste also für jeden Nutzer mehrfach übertragen werden. Bei einem entsprechend großen Kundenstamm entstünden in der Praxis durch dieses Modell erhebliche Datenmengen. 

Um dieses Problem zu umgehen werden Nutzer in Gruppen eingeteilt die jeweils einen gemeinsamen Gruppenschlüssel haben. Dieser wird mit dem individuellen $PK_{i} $verschlüsselt und an alle Nutzer in der Gruppe verteilt. Anschließend wird der Gruppenschlüssel verwendet um SK zu verschlüsseln. Der GK ändert sich nur, wenn ein Mitglied die Gruppe verlässt. Bei geschickter Wahl der Gruppen, kann das Datenaufkommen durch das Ausscheiden von Nutzern erheblich reduziert werden.

\end{document}
