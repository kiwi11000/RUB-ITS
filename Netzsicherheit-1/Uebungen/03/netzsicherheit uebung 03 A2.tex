\documentclass[10pt,a4paper]{article}
\usepackage[utf8x]{inputenc}
\usepackage{ucs}
\usepackage{amsmath}
\author{Tilman Bender   Matrikelnummer: 108011247244\\Christian Kröger,   Matrikelnummer: 108011250663\\Thomas Tacke  Matrikelnummer: 108011267882}
\title{Netzwerksicherheit Übung 3 Aufgabe 2}

%\fancyfoot[L]{Christian Kröger}
\begin{document}
\maketitle

\section*{Aufgabe 2}
\subsection*{2.1}
Die Angreifbarkeit von Microsofts Implementierung von PPTP basiert auf den Designschwächen des LAN-Manager Hashes (LMH) und dem bei MS-CHAP eingesetzten Verfahren zum Berechnen der Response.
Designschwächen im LAN-Manager Hash:
\begin{enumerate}
\item Verringerung des Schlüsselraumes durch Begrenzung der Länge auf 14 Zeichen (Gegenmaßnahme: Zulassen beliebiger Länge)
\item Verringerung des Schlüsselraumes durch Konvertierung des Passworts in Großbuchstaben (Gegenmaßnahme: Konvertierung unterlassen)
\item Auffüllen zu kurzer Passwörter mit Nullen und separates Verschlüsseln führt zu leichter Erkennbarkeit von Passwörtern mit 7 oder weniger Zeichen (Gegenmaßnahme: Padding \& Auftrennung unterlassen)
\item Aufteilen des Passworts und separates Verschlüsseln mit S erlaubt separate Angriffe mit kleineren Wörterbüchern (Gegenmaßnahme: Passwort als ganzes hashen)
\item Es werden keine Salts verwendet. Dies erlaubt die Erstellung von wiederverwendbaren Hash-Dictionaries. (Gegenmaßnahme: Salts verwenden)
\end{enumerate}
Microsoft hat einen Großteil dieser Schwächen im Windows NT Hash behoben.
Designschwächen von MS-CHAP:
\begin{enumerate}
	\item Zero-Padding und Aufspalten des Schlüssels erlaubt Rekonstruktion der letzten zwei Bytes des Schlüssels
	\item Da der Schlüssel durch LMH erzeugt ist und somit dessen Schwächen gelten, kann das Passwort rekonstruiert werden.
\end{enumerate}
Gegenmaßnahme: Auf LMH gänzlich verzichten und nur noch WinNT-Hash verwenden. Eine Berechnung der letzten beiden Schlüssel-Bytes (H14,H15 ist immer noch Möglich, führt aber nicht mehr zur Kompromittierung des Passwortes.

\section*{2.2}
\begin{enumerate}
\item Probiere alle möglichen Werte von $H_{7}$ . Der richtige Wert ist zu erkennen, wenn C mit $H_{7}$,0 0,0,0,0,0 mit R7..15 übereinstimmt. Dauer: $2^{8}$ Operationen
\item Probiere die Werte von  $P_{0},\dots, P_{6}$ Falsche Werte können durch die Verschlüsselung  der Konstanten S  und Abgleich des letzten Bytes des Chiffrats mit dem zuvor berechneten Wert für  H7 aussortiert werden.
\item Von Verbleibenden Kandidaten für $P_{0},\dots, P_{6}$ werden die Hashes dazu verwendet um die Challenge mittels DES zu verschlüsseln. Der Kandidat, der mit $R_{0}...R_{7}$ übereinstimmt bleibt übrig.
\item Berechne die WinNT-Hashes aller Kombinationen von Groß-/Kleinschreibung des Kandidaten aus dem vorigen Schritt und berechne die Responses. Vergleiche diese im den tatsächlichen WinNT-Hash Responses
\end{enumerate}

\end{document}