\documentclass[12pt.twoside,a4paper,notitlepage,parskip]{scrartcl}
\usepackage[utf8x]{inputenc}
\usepackage{ucs}
\usepackage{ngerman}
\usepackage{amsmath}
\usepackage{amsfonts}
\usepackage{amssymb}
\usepackage{nameref}
\usepackage{enumerate}

\begin{document}
\title{Netzsicherheit I, WS 2011, Übung 6}
\author{
\begin{tabular}{ccc}
Tilman Bender & Christian Kröger & Thomas Tacke \\
108011247244 & 108011250663 & 108011267882 \\
\end{tabular}
}
\date{\today}
\maketitle

\section{Broadcast Encryption III}
\begin{align*}
U_{1}:(N,p_{1},g_{1})&=(179141,p_{3}*7,?)\\
U_{2}:(N,p_{2},g_{2})&=(179141,p_{2},?)\\
U_{3}:(N,p_{3},g_{3})&=(179141,7,67050)
\end{align*}

Da $p_{3}$ ein Teiler von $p_{1}$ ist, sind die Werte von $g_{1}$ und $g_{3}$  identisch:
\begin{align*}
g_{1}&=g^{p_{3}*x}=g^{7*x} \mod{N}\\
g_{3}&=g^{p_{3}}=g^{7} \mod{N}\\
\end{align*} 
Da uns als Angreifer $g_{3}$ bekannt ist, kennen wir nun auch Geheimnis  $g_{1}$ eines der Gruppenmitglieder. 
Daraus kann der Gruppenschlüssel berechnet werden.

\begin{align*}
g_{p_{T}}&=g_{1}^{p_{2}}=g_{3}^{p_{2}} \mod{N}\\
g_{p_{T}}&=138791
\end{align*} 

\end{document}
