\documentclass[12pt.twoside,a4paper,notitlepage]{article}
\usepackage[utf8x]{inputenc}
\usepackage{ucs}
\usepackage{ngerman}
\usepackage{amsmath}
\usepackage{amsfonts}
\usepackage{amssymb}
\usepackage{nameref}
\usepackage{enumerate}

\begin{document}
\title{Netzsicherheit I, WS 2011, Übung 2}
\author{
\begin{tabular}{ccc}
Tilman Bender & Christian Kröger & Thomas Tacke \\
108011247244 & 108011250663 & 108011267882 \\
\end{tabular}
}
\date{\today}
\maketitle

\section*{WLAN \& RC4}
\begin{enumerate}[a)]
\item Wie man anhand der Tabelle \nameref{tab:ksa} sieht, ist der Output nicht stimmig.\\
$K[5] = 10 \wedge \neq 7$
TODO: Yaddda Yadda dings da.
\begin{table}[h]
\centering
\begin{tabular}{|c|c||c|c|c|c|c|c|c|c|c|c|c|c|c|c|c|c|}
\hline
i & j & 0 & 1 & 2 & 3 & 4 & 5 & 6 & 7 & 8 & 9 & 10 & 11 & 12 & 13 & 14 & 15 \\
\hline
0 & 4 & 4 & 1 & 2 & 3 & 0 & 5 & 6 & 7 & 8 & 9 & 10 & 11 & 12 & 13 & 14 & 15 \\
\hline
1 & 3 & 4 & 3 & 2 & 1 & 0 & 5 & 6 & 7 & 8 & 9 & 10 & 11 & 12 & 13 & 14 & 15 \\
\hline
2 & 14 & 4 & 3 & 14 & 1 & 0 & 5 & 6 & 7 & 8 & 9 & 10 & 11 & 12 & 13 & 2 & 15 \\
\hline
3 & 12 & 4 & 3 & 14 & 12 & 0 & 5 & 6 & 7 & 8 & 9 & 10 & 11 & 1 & 13 & 2 & 15 \\
\hline
4 & 7 & 4 & 3 & 14 & 12 & 7 & 5 & 6 & 0 & 8 & 9 & 10 & 11 & 1 & 13 & 2 & 15 \\
\hline
5 & 3 & 4 & 3 & 14 & 5 & 7 & 12 & 6 & 0 & 8 & 9 & 10 & 11 & 1 & 13 & 2 & 15 \\
\hline
\end{tabular}
\caption{Hilfsmittel zur KSA-Berechnung a)}
\label{tab:ksa}
\end{table}

\item $K[6=13]$\\
Yadda Yadda dings da TODO.
\begin{table}[h]
\centering
\begin{tabular}{|c|c||c|c|c|c|c|c|c|c|c|c|c|c|c|c|c|c|}
\hline
i & j & 0 & 1 & 2 & 3 & 4 & 5 & 6 & 7 & 8 & 9 & 10 & 11 & 12 & 13 & 14 & 15 \\
\hline
0 & 4 & 4 & 1 & 2 & 3 & 0 & 5 & 6 & 7 & 8 & 9 & 10 & 11 & 12 & 13 & 14 & 15 \\
\hline
1 & 3 & 4 & 3 & 2 & 1 & 0 & 5 & 6 & 7 & 8 & 9 & 10 & 11 & 12 & 13 & 14 & 15 \\
\hline
2 & 14 & 4 & 3 & 14 & 1 & 0 & 5 & 6 & 7 & 8 & 9 & 10 & 11 & 12 & 13 & 2 & 15 \\
\hline
3 & 12 & 4 & 3 & 14 & 12 & 0 & 5 & 6 & 7 & 8 & 9 & 10 & 11 & 1 & 13 & 2 & 15 \\
\hline
4 & 7 & 4 & 3 & 14 & 12 & 7 & 5 & 6 & 0 & 8 & 9 & 10 & 11 & 1 & 13 & 2 & 15 \\
\hline
5 & 6 & 4 & 3 & 14 & 12 & 7 & 6 & 5 & 0 & 8 & 9 & 10 & 11 & 1 & 13 & 2 & 15 \\
\hline
6 & 8 & 4 & 3 & 14 & 12 & 7 & 6 & 8 & 0 & 5 & 9 & 10 & 11 & 1 & 13 & 2 & 15 \\
\hline
\end{tabular}
\caption{Hilfsmittel zur KSA-Berechnung b)}
\label{tab:ksa}
\end{table}

\item Yadda Yadde dings da TODO\\
wharscheinlichkeiten $\approx  60$ dingsdas, weist schon qed.

\end{enumerate}



\end{document}
