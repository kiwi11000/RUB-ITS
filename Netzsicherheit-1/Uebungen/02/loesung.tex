\documentclass[12pt.twoside,a4paper,notitlepage]{article}
\usepackage[utf8x]{inputenc}
\usepackage{ucs}
\usepackage{ngerman}
\usepackage{amsmath}
\usepackage{amsfonts}
\usepackage{amssymb}
\usepackage{nameref}
\usepackage{enumerate}

\begin{document}
\title{Netzsicherheit I, WS 2011, Übung 2}
\author{
\begin{tabular}{ccc}
Tilman Bender & Christian Kröger & Thomas Tacke \\
108011247244 & 108011250663 & 108011267882 \\
\end{tabular}
}
\date{\today}
\maketitle

\section*{WLAN \& RC4}
\begin{enumerate}[a)]
\item Wie man anhand der Tabelle \nameref{tab:ksa} sieht, funktioniert der Angriff nicht, da der Output nicht stimmig ist.\\
Dies ist daran zu sehen, das die 5 der erste z Wert ist, und dieser daher in der nächste Operation an die Stelle bewegt werden müsste,
auf die durch S[S[1]+S[S[1]] verwiesen wird. In diesem Fall wäre dies die Position 8.\\ Die 5 kann jedoch nicht auf diese
Position bewegt werden, da i auf Position 6 zeigt, was bedeutet das ein beliebiges Element mit dem an Position 6 getauscht werden 
kann jedoch nicht mit dem an Position 8 (bzw. es ist nicht möglich Position 3 mit 8 zu tauschen, was nötig wäre um die 5 auf 
Position 8 zu verschieben, da eine der beiden zu vertauschenden Positionen aufjedenfall Position 6 ist).\\
Um es zu ermöglichen, das im nächsten Schritt die Position der 5 verändert werden kann, ist es nötig diese auf Position 6 zu 
verschieben (denn dadurch wird i im folgenden Schritt auf die 5 verweisen und es ist möglich die 5 mit Hilfe des zu berechnenden $K[6]$
an die korrekte Position zu verschieben).\\
Die 5 kann durch folgende Änderung auf Position 6 verschoben werden: \\
$K[5] = 10 \wedge \neq 7$\\
Sprich, wenn $K[5]$ auf den Wert 10 gesetzt wird, steht der Wert 5 auf Position 6 und kann durch $K[6]$ im nächsten 
Schritt an die benötigte Position verschoben werden.
%TODO: Yaddda Yadda dings da.
%daher so zurück das erster output = 5 bzw. das mit dem nächsten keybit 5 auf die korrekte stelle gesetzt werden kann.
%deshalb k5 zu 10 machen damit im nächsten schritt die s[6] auf 5 zeigt und die 5 an die richtige position getauscht werden kann.
\begin{table}[h]
\centering
\begin{tabular}{|c|c||c|c|c|c|c|c|c|c|c|c|c|c|c|c|c|c|}
\hline
i & j & 0 & 1 & 2 & 3 & 4 & 5 & 6 & 7 & 8 & 9 & 10 & 11 & 12 & 13 & 14 & 15 \\
\hline
0 & 4 & 4 & 1 & 2 & 3 & 0 & 5 & 6 & 7 & 8 & 9 & 10 & 11 & 12 & 13 & 14 & 15 \\
\hline
1 & 3 & 4 & 3 & 2 & 1 & 0 & 5 & 6 & 7 & 8 & 9 & 10 & 11 & 12 & 13 & 14 & 15 \\
\hline
2 & 14 & 4 & 3 & 14 & 1 & 0 & 5 & 6 & 7 & 8 & 9 & 10 & 11 & 12 & 13 & 2 & 15 \\
\hline
3 & 12 & 4 & 3 & 14 & 12 & 0 & 5 & 6 & 7 & 8 & 9 & 10 & 11 & 1 & 13 & 2 & 15 \\
\hline
4 & 7 & 4 & 3 & 14 & 12 & 7 & 5 & 6 & 0 & 8 & 9 & 10 & 11 & 1 & 13 & 2 & 15 \\
\hline
5 & 3 & 4 & 3 & 14 & 5 & 7 & 12 & 6 & 0 & 8 & 9 & 10 & 11 & 1 & 13 & 2 & 15 \\
\hline
\end{tabular}
\caption{Hilfsmittel zur KSA-Berechnung a)}
\label{tab:ksa}
\end{table}

\item $K[6=4]$\\
Dies ist zu sehen durch: \\
$z[0]=5=S[S[1]+S[S[1]]]=S[3+S[3]]=S[3+12]=S[15]$. Daher muss an Position 15 der Wert 5 stehen, was durch $K[6]$ zu 
erreichen ist.\\
%also output an s[15] muss gleich 5 sein\\
%Yadda Yadda dings da TODO. \\
Dies führt zu der Berechnung: \\
$j = (j + S[i] + k[6]) \mod 16 $\\
$j = 6 $ (alter j Wert) $+ 5 + k[6]$ \\
$15 = 11 + k[6]$\\
$K[6] = 4 $
\begin{table}[h]
\centering
\begin{tabular}{|c|c||c|c|c|c|c|c|c|c|c|c|c|c|c|c|c|c|}
\hline
i & j & 0 & 1 & 2 & 3 & 4 & 5 & 6 & 7 & 8 & 9 & 10 & 11 & 12 & 13 & 14 & 15 \\
\hline
0 & 4 & 4 & 1 & 2 & 3 & 0 & 5 & 6 & 7 & 8 & 9 & 10 & 11 & 12 & 13 & 14 & 15 \\
\hline
1 & 3 & 4 & 3 & 2 & 1 & 0 & 5 & 6 & 7 & 8 & 9 & 10 & 11 & 12 & 13 & 14 & 15 \\
\hline
2 & 14 & 4 & 3 & 14 & 1 & 0 & 5 & 6 & 7 & 8 & 9 & 10 & 11 & 12 & 13 & 2 & 15 \\
\hline
3 & 12 & 4 & 3 & 14 & 12 & 0 & 5 & 6 & 7 & 8 & 9 & 10 & 11 & 1 & 13 & 2 & 15 \\
\hline
4 & 7 & 4 & 3 & 14 & 12 & 7 & 5 & 6 & 0 & 8 & 9 & 10 & 11 & 1 & 13 & 2 & 15 \\
\hline
5 & 6 & 4 & 3 & 14 & 12 & 7 & 6 & 5 & 0 & 8 & 9 & 10 & 11 & 1 & 13 & 2 & 15 \\
\hline
6 & 15 & 4 & 3 & 14 & 12 & 7 & 6 & 15 & 0 & 8 & 9 & 10 & 11 & 1 & 13 & 2 & 5 \\
\hline
\end{tabular}
\caption{Hilfsmittel zur KSA-Berechnung b)}
\label{tab:ksa}
\end{table}

\item %Yadda Yadde dings da TODO\\
%wharscheinlichkeiten $\approx  60$ dingsdas, weist schon qed.
%Daher krasser stuff der nich immer klappt

Die Berechnung des Schlüsselwortes ist nicht immer korrekt, was daran gesehen werden kann, das im Durchschnitt ca. 60
verschiedene Werte benötigt werden um ein folgendes Schlüssel Element (von K) zu errechnen.\\
Die Wahrscheinlichkeit, das dies für einen passenden IV funktioniert ist ca. 5 Prozent, daher ist man mit ca. 60 Werten jenseits der 50 Prozent.\\

\end{enumerate}



\end{document}
