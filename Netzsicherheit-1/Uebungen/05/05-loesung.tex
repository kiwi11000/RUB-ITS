\documentclass[12pt.twoside,a4paper,notitlepage,parskip]{scrartcl}
\usepackage[utf8x]{inputenc}
\usepackage{ucs}
\usepackage{ngerman}
\usepackage{amsmath}
\usepackage{amsfonts}
\usepackage{amssymb}
\usepackage{nameref}
\usepackage{enumerate}

\begin{document}
\title{Netzsicherheit I, WS 2011, Übung 5}
\author{
\begin{tabular}{ccc}
Tilman Bender & Christian Kröger & Thomas Tacke \\
108011247244 & 108011250663 & 108011267882 \\
\end{tabular}
}
\date{\today}
\maketitle

\section{Broadcast Encryption I}

\section{Broadcast Encryption II}
\subsection{a)} Öffentlich ist der Moduls $N=PQ$ (629) sowie die Zuordnung der einzelnen Primzahlen zu den Nutzern $((U_{1},7), (U_{2},5), (U_{3},11))$. Geheim dagegen sind ist die Basis $g \in \mathbb{Z}_{m}$, die Faktoren P, Q sowie die jeweiligen geheimen Zahlen $g_{i}$ (133, 92, 5). 

\subsection{b)} 
\begin{align*}
K_{1}&=133^{5*11} \mod{629}&= 533\\
K_{2}&=92^{7*11} \mod{629}&= 533\\
K_{2}&=5^{5*7} \mod{629}&= 533
\end{align*}
\subsection{c)}
\begin{align*}
K_{1}&=133^{11}\mod{629}&=129\\
K_{2}&=5^{7} \mod{629}&= 129\\
\end{align*}
\subsection{d)}
\end{document}
