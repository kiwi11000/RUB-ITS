\documentclass[10pt,a4paper]{article}
\usepackage[utf8x]{inputenc}
\usepackage{ucs}
\usepackage{ngerman}
\usepackage{amsmath}
\usepackage{amsfonts}
\usepackage{amssymb}
\usepackage{fancyhdr}

\author{Tilman Bender   Matrikelnummer: 108011247244\\}
\title{Einführung in Kryptographie und Datensicherheit Übung 7}

\begin{document}
\maketitle

\section*{Aufgabe 1}

\subsection*{a)}
$GF(2)=GF(p^{m})=GF(2^{1})$ Die Polynome haben die Form 
\begin{align*}
A_{1}(x)=0*x^{0}\\
A_{2}(x)=1*x^{0}
\end{align*}
Es gibt $2^{1}$ Polynome A(x).
\subsection*{b)}
Es existiert nur dann ein GF(n), wenn sich n als Potenz einer Primzahl darstellen lässt ($p^{m}$). Das Polynom A(x) soll maximal den Grad $m-1$ haben. Die $m$ einzelnen Glieder des Polynoms haben also die Werte $x^{0} \dots x^{m-1}$. 
\begin{align*}
A(x)=\sum\limits_{j=0}^{m-1} a_{i}*x^{j}, a_{i} \in \{0 \dots p-1\}\\
\end{align*}
Jeder der $m$ Koeffizienten  $a_i$ kann $p$ verschiedene Werte ($0 \dots p-1$) annehmen. Es existieren also $p^m=n$ verschiedene Polynome A(x).

\end{document}