\documentclass[10pt,a4paper]{article}
\usepackage[utf8x]{inputenc}
\usepackage{ucs}
\usepackage{ngerman}
\usepackage{amsmath}
\usepackage{amsfonts}
\usepackage{amssymb}
\usepackage{fancyhdr}

\author{Tilman Bender   Matrikelnummer: 108011247244\\}
\title{Einführung in Kryptographie und Datensicherheit Übung 6}

\begin{document}
\maketitle

\section*{Aufgabe 2}
\subsection*{a)}
Der einzige Unterschied zwischen der Verschlüsselung und der Entschlüsselung von DES ist, dass bei der Entschlüsselung der Schlüsselfahrplan umgekehrt ($K_{16} \dots K_{1}$) durchlaufen wird.

\subsection*{b)}
\begin{align*}
L_{1}^{d}&=R_{0}^{d}\\
R_{1}^{d}&=L_{0}^{d} \oplus f(R_{0}^{d},K_{16})\\
\\
L_{1}^{d}&=L_{16}^{e}\\
R_{1}^{d}&=R_{16}^{e} \oplus f(L_{16}^{e},K_{16})\\
\\
L_{1}^{d}&=R_{15}^{e}\\
R_{1}^{d}&=L_{15}^{e} \oplus f(R_{15}^{e},K_{16})  \oplus f(R_{15}^{e},K_{16})\\
\\
L_{1}^{d}&=R_{15}^{e}\\
R_{1}^{d}&=L_{15}^{e}
\end{align*}
Der Text nach der ersten Runde der Entschlüsselung  ist also der gleiche wie vor der sechzehnten Runde der Verschlüsselung.
Die erneute Anwendung der F-Funktion mit $K_{16}$ auf das Ergebnis der Verschlüsselung führt also zu einer Umkehrung.
\end{document}