\documentclass[10pt,a4paper,parskip]{scrartcl}
\usepackage[utf8x]{inputenc}
\usepackage{ucs}
\usepackage{ngerman}
\usepackage{amsmath}
\usepackage{amsfonts}
\usepackage{amssymb}
\usepackage{fancyhdr}
%fuer Unterstreichungen
\usepackage{ulem}

\author{Tilman Bender   Matrikelnummer: 108011247244\\}
\title{Einführung in Kryptographie und Datensicherheit Übung 5}

\begin{document}
\maketitle

\section*{Aufgabe 4}
\subsection*{a)}
\begin{enumerate}
\item Es fehlt eine hexadezimale Stelle (4 Bit). Anfänglich ist der max. Suchraum=$2^{4}$
\item Das fehlende Symbol war eine Zahl. Suchraum reduziert sich auf 0000-1001 = 10 Kombinationen
\item Das vierte fehlende Bit hat auf die Entschlüsselung keinen Einfluss. Man muss also nur 000 - 100 permutieren. = $2^{3}$ = 8 Kombinationen
\end{enumerate}

Im Durchschnitt muss man also $\frac{2^{3}}{2}$ und im schlechtesten Fall $2^{3}$ Kombinationen durchprobieren.

\subsection*{b)}
i)
\sout{0000}, \sout{0001}, \sout{0010}, \sout{0011}\\
$k_{60\dots64}=0100$ (da das Parity-Bit keine Beachtung findet wäre 0101 ebenfalls eine mögliche Lösung)

ii)
\texttt{{\small Du kannst es dir sparen bei Aufgabe 5.4 vom aktuellen Kryptozettel die ganzen Schlüssel durchzuprobieren.
Ich habe zufällig bei M.Duermuth im Büro gesehen, dass der Schlüssel mit der Zahl 4 ergänzt
werden muss. Also nimm einfach B3 3A 89 2B A5 2B CC F4 und ECB Mode vom DES, dann bist
du schön schnell fertig.}}

\end{document}