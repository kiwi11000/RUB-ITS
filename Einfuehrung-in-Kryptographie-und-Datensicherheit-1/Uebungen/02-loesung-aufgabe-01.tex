\documentclass[10pt,a4paper]{article}
\usepackage[utf8x]{inputenc}
\usepackage{ucs}
\usepackage{ngerman}
\usepackage{amsmath}
\usepackage{amsfonts}
\usepackage{amssymb}
\usepackage{fancyhdr}

\author{Tilman Bender   Matrikelnummer: 108011247244\\}
\title{Einführung in Kryptographie und Datensicherheit Übung 2}

\begin{document}
\maketitle

\section*{Aufgabe 1}
\subsection*{a)}
\renewcommand{\theequation}{\roman{equation}}
\begin{align} 
13*24 \equiv 22 \mod 29 
\end{align}
\begin{align} 
17 * 1337 \mod 29 \notag\\
17 * 3 \equiv 22 \mod 29 \end
{align}
\begin{align} 
69 * 31\mod 29 \notag \\
11 * 2  \equiv 22 \mod 29 \end{align}
\begin{align}  
(-36)*(-28) \mod 29 \notag \\
22 * 1 \equiv 22 \mod 29 \end{align}
\begin{align}
231 * (-51) \mod29  \notag \\
-1 * 7 \equiv 22 \mod 29
 \end{align}
\subsection*{b)}
In allen fünf Fällen können die Multiplikationen vereinfacht werden indem man kleinere Faktoren aus der gleichen Äquivalenzklasse verwendet.
\subsection*{c)}
\setcounter{equation}{0}
\begin{equation}  3 * 3^{-1} \equiv 1 \Rightarrow 3^{-1} \equiv 10 \mod 29 \end{equation}
\begin{align}   6 * 6^{-1} \equiv 1 \Rightarrow 6^{-1} \equiv 4 \mod 23 \notag\\
 \Rightarrow 2 * 6^{-1} \equiv 8 \mod 23 \end{align}
\begin{align}   
3*3^{-1} \equiv 1\Rightarrow 3^{-1} \equiv 9 \mod 13 \notag \\
9*9^{-1} \equiv 1 \Rightarrow 9^{-1} \equiv  3 \mod 13\\
\Rightarrow 7*3^{-1}*4*9^{-1} \equiv 2 \mod 13 \notag
\end{align}
\end{document}